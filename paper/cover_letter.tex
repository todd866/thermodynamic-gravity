\documentclass[11pt]{letter}
\usepackage[margin=1in]{geometry}
\usepackage{hyperref}

\signature{Ian Todd\\Sydney Medical School\\University of Sydney}
\address{Sydney Medical School\\University of Sydney\\Sydney, NSW, Australia\\itod2305@uni.sydney.edu.au}

\begin{document}

\begin{letter}{Editorial Office\\Foundations of Physics}

\opening{Dear Editors,}

I am pleased to submit ``Gravity as Constraint Compliance: A Thermodynamic Framework for Quantum Gravity'' for consideration in Foundations of Physics.

\textbf{Summary.} We reinterpret Jacobson's local-horizon Clausius relation as a local admissibility constraint rather than emergent thermodynamics. The quantized subsystem is the constraint interface itself---not the metric and not unspecified ``horizon microstates.'' This yields a distinct stochastic noise channel (certification fluctuations, independent of stress-energy correlators) and organizes non-equilibrium corrections as interface dissipation rather than ``thermodynamics of spacetime.''

\textbf{Key contributions:}
\begin{enumerate}
\item \textbf{Reinterpretation of thermodynamic gravity.} The Clausius relation on local horizons functions as an admissibility condition, not emergent thermodynamics. Einstein's equation emerges as the compatibility condition across all such interfaces.

\item \textbf{Localization of quantumness.} Quantum gravitational effects arise from finite resolution and fluctuations at constraint interfaces---proposing an answer to the question of what should be quantized in thermodynamic gravity.

\item \textbf{Controlled corrections.} The framework predicts stochastic corrections (from certification fluctuations) and higher-curvature corrections (from entropy production), with explicit scaling.

\item \textbf{Interpretation of $\hbar$.} Planck's constant sets the certification resolution scale---the minimum distinguishable area increment at constraint interfaces. The scaling is expressed as $G\hbar/A$, so $\hbar$ cancels cleanly with the Bekenstein-Hawking coefficient $\eta = 1/(4G\hbar)$ to yield $\Delta S_{\min} \sim \mathcal{O}(1)$.
\end{enumerate}

\noindent The novelty relative to Jacobson and Padmanabhan is localizing quantumness to the interface layer and deriving quantum corrections from certification limits, rather than treating horizon thermodynamics as emergent from unspecified microstates. The manuscript connects explicitly to stochastic gravity and non-equilibrium thermodynamic gravity, grounding the framework in established literature. It includes explicit scaling predictions for certification-noise amplitude ($\sim \kappa^3 \ell_P^2/A$) and the derivative-expansion form of non-equilibrium corrections.

\textbf{Why Foundations of Physics.} The mathematical backbone follows Jacobson, but the contribution is a clarified ontology of what is being ``thermodynamically counted'' and what is being quantized. This reorganizes known correction mechanisms and adds a distinct interface-noise channel---state-dependent focusing variance not tied to stress-energy correlators. For a journal focused on foundational questions in physics, this addresses a central ambiguity in the thermodynamic gravity program: what should be quantized, and what are the consequences of that choice?

The manuscript contains seven explicit postulates, a formal definition of certification (with the Clausius residual $\Delta_\Sigma$ as central object), and a comparison table situating the framework relative to Jacobson, Eling--Jacobson, and Einstein--Langevin stochastic gravity. It has not been submitted elsewhere.

\closing{Sincerely,}

\end{letter}
\end{document}
